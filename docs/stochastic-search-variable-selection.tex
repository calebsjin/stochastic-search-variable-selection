\documentclass[]{book}
\usepackage{lmodern}
\usepackage{amssymb,amsmath}
\usepackage{ifxetex,ifluatex}
\usepackage{fixltx2e} % provides \textsubscript
\ifnum 0\ifxetex 1\fi\ifluatex 1\fi=0 % if pdftex
  \usepackage[T1]{fontenc}
  \usepackage[utf8]{inputenc}
\else % if luatex or xelatex
  \ifxetex
    \usepackage{mathspec}
  \else
    \usepackage{fontspec}
  \fi
  \defaultfontfeatures{Ligatures=TeX,Scale=MatchLowercase}
\fi
% use upquote if available, for straight quotes in verbatim environments
\IfFileExists{upquote.sty}{\usepackage{upquote}}{}
% use microtype if available
\IfFileExists{microtype.sty}{%
\usepackage{microtype}
\UseMicrotypeSet[protrusion]{basicmath} % disable protrusion for tt fonts
}{}
\usepackage{hyperref}
\hypersetup{unicode=true,
            pdftitle={Approaches for Bayesian Variable Selection (SSVS)},
            pdfauthor={Shiqiang Jin},
            pdfborder={0 0 0},
            breaklinks=true}
\urlstyle{same}  % don't use monospace font for urls
\usepackage{natbib}
\bibliographystyle{apalike}
\usepackage{color}
\usepackage{fancyvrb}
\newcommand{\VerbBar}{|}
\newcommand{\VERB}{\Verb[commandchars=\\\{\}]}
\DefineVerbatimEnvironment{Highlighting}{Verbatim}{commandchars=\\\{\}}
% Add ',fontsize=\small' for more characters per line
\usepackage{framed}
\definecolor{shadecolor}{RGB}{248,248,248}
\newenvironment{Shaded}{\begin{snugshade}}{\end{snugshade}}
\newcommand{\AlertTok}[1]{\textcolor[rgb]{0.94,0.16,0.16}{#1}}
\newcommand{\AnnotationTok}[1]{\textcolor[rgb]{0.56,0.35,0.01}{\textbf{\textit{#1}}}}
\newcommand{\AttributeTok}[1]{\textcolor[rgb]{0.77,0.63,0.00}{#1}}
\newcommand{\BaseNTok}[1]{\textcolor[rgb]{0.00,0.00,0.81}{#1}}
\newcommand{\BuiltInTok}[1]{#1}
\newcommand{\CharTok}[1]{\textcolor[rgb]{0.31,0.60,0.02}{#1}}
\newcommand{\CommentTok}[1]{\textcolor[rgb]{0.56,0.35,0.01}{\textit{#1}}}
\newcommand{\CommentVarTok}[1]{\textcolor[rgb]{0.56,0.35,0.01}{\textbf{\textit{#1}}}}
\newcommand{\ConstantTok}[1]{\textcolor[rgb]{0.00,0.00,0.00}{#1}}
\newcommand{\ControlFlowTok}[1]{\textcolor[rgb]{0.13,0.29,0.53}{\textbf{#1}}}
\newcommand{\DataTypeTok}[1]{\textcolor[rgb]{0.13,0.29,0.53}{#1}}
\newcommand{\DecValTok}[1]{\textcolor[rgb]{0.00,0.00,0.81}{#1}}
\newcommand{\DocumentationTok}[1]{\textcolor[rgb]{0.56,0.35,0.01}{\textbf{\textit{#1}}}}
\newcommand{\ErrorTok}[1]{\textcolor[rgb]{0.64,0.00,0.00}{\textbf{#1}}}
\newcommand{\ExtensionTok}[1]{#1}
\newcommand{\FloatTok}[1]{\textcolor[rgb]{0.00,0.00,0.81}{#1}}
\newcommand{\FunctionTok}[1]{\textcolor[rgb]{0.00,0.00,0.00}{#1}}
\newcommand{\ImportTok}[1]{#1}
\newcommand{\InformationTok}[1]{\textcolor[rgb]{0.56,0.35,0.01}{\textbf{\textit{#1}}}}
\newcommand{\KeywordTok}[1]{\textcolor[rgb]{0.13,0.29,0.53}{\textbf{#1}}}
\newcommand{\NormalTok}[1]{#1}
\newcommand{\OperatorTok}[1]{\textcolor[rgb]{0.81,0.36,0.00}{\textbf{#1}}}
\newcommand{\OtherTok}[1]{\textcolor[rgb]{0.56,0.35,0.01}{#1}}
\newcommand{\PreprocessorTok}[1]{\textcolor[rgb]{0.56,0.35,0.01}{\textit{#1}}}
\newcommand{\RegionMarkerTok}[1]{#1}
\newcommand{\SpecialCharTok}[1]{\textcolor[rgb]{0.00,0.00,0.00}{#1}}
\newcommand{\SpecialStringTok}[1]{\textcolor[rgb]{0.31,0.60,0.02}{#1}}
\newcommand{\StringTok}[1]{\textcolor[rgb]{0.31,0.60,0.02}{#1}}
\newcommand{\VariableTok}[1]{\textcolor[rgb]{0.00,0.00,0.00}{#1}}
\newcommand{\VerbatimStringTok}[1]{\textcolor[rgb]{0.31,0.60,0.02}{#1}}
\newcommand{\WarningTok}[1]{\textcolor[rgb]{0.56,0.35,0.01}{\textbf{\textit{#1}}}}
\usepackage{longtable,booktabs}
\usepackage{graphicx,grffile}
\makeatletter
\def\maxwidth{\ifdim\Gin@nat@width>\linewidth\linewidth\else\Gin@nat@width\fi}
\def\maxheight{\ifdim\Gin@nat@height>\textheight\textheight\else\Gin@nat@height\fi}
\makeatother
% Scale images if necessary, so that they will not overflow the page
% margins by default, and it is still possible to overwrite the defaults
% using explicit options in \includegraphics[width, height, ...]{}
\setkeys{Gin}{width=\maxwidth,height=\maxheight,keepaspectratio}
\IfFileExists{parskip.sty}{%
\usepackage{parskip}
}{% else
\setlength{\parindent}{0pt}
\setlength{\parskip}{6pt plus 2pt minus 1pt}
}
\setlength{\emergencystretch}{3em}  % prevent overfull lines
\providecommand{\tightlist}{%
  \setlength{\itemsep}{0pt}\setlength{\parskip}{0pt}}
\setcounter{secnumdepth}{5}
% Redefines (sub)paragraphs to behave more like sections
\ifx\paragraph\undefined\else
\let\oldparagraph\paragraph
\renewcommand{\paragraph}[1]{\oldparagraph{#1}\mbox{}}
\fi
\ifx\subparagraph\undefined\else
\let\oldsubparagraph\subparagraph
\renewcommand{\subparagraph}[1]{\oldsubparagraph{#1}\mbox{}}
\fi

%%% Use protect on footnotes to avoid problems with footnotes in titles
\let\rmarkdownfootnote\footnote%
\def\footnote{\protect\rmarkdownfootnote}

%%% Change title format to be more compact
\usepackage{titling}

% Create subtitle command for use in maketitle
\providecommand{\subtitle}[1]{
  \posttitle{
    \begin{center}\large#1\end{center}
    }
}

\setlength{\droptitle}{-2em}

  \title{Approaches for Bayesian Variable Selection (SSVS)}
    \pretitle{\vspace{\droptitle}\centering\huge}
  \posttitle{\par}
    \author{Shiqiang Jin}
    \preauthor{\centering\large\emph}
  \postauthor{\par}
      \predate{\centering\large\emph}
  \postdate{\par}
    \date{4-17-2017}

\usepackage{booktabs}

\begin{document}
\maketitle

{
\setcounter{tocdepth}{1}
\tableofcontents
}
\hypertarget{foreword}{%
\chapter{Foreword}\label{foreword}}

I am \href{https://www.sjin.name/}{Caleb Jin}. After I read this paper, \textbf{Approaches for Bayesian Variable Selection (SSVS)}\citep{George1997} and \citep{George1993}, I write down the nodes of the key idea and R code to realize it.

\newcommand\T{{\top}}
\newcommand\ubeta{{\boldsymbol \beta}}
\newcommand\uSigma{{\boldsymbol \Sigma}}
\newcommand\uepsilon{{\boldsymbol \epsilon}}
\newcommand\umu{{\boldsymbol \mu}}
\newcommand\utheta{{\boldsymbol \theta}}
\newcommand\bg{{\boldsymbol \gamma}}
\newcommand\0{{\bf 0}}
\newcommand\uX{{\bf X}}
\newcommand\uD{{\bf D}}
\newcommand\ux{{\bf x}}
\newcommand\uY{{\bf Y}}
\newcommand\uy{{\bf y}}
\newcommand\uz{{\bf z}}
\newcommand\uI{{\bf I}}
\newcommand\uA{{\bf A}}
\newcommand\uB{{\bf B}}
\newcommand\uH{{\bf H}}
\newcommand\uM{{\bf M}}
\newcommand\uV{{\bf V}}
\newcommand\diag{{\rm diag}}

\hypertarget{approaches-for-bayesian-variable-selection-ssvs}{%
\chapter{Approaches for Bayesian Variable Selection (SSVS)}\label{approaches-for-bayesian-variable-selection-ssvs}}

Consider a high dimensional linear regression model as follows:
\begin{eqnarray}
{\bf y}={\bf X}{\boldsymbol \beta}+ {\boldsymbol \epsilon}
\label{eq:1}
\end{eqnarray}
where \({\bf y}=(y_1,y_2,\ldots,y_{n})^{{\top}}\) is the \(n\)-dimensional response vector, \({\bf X}=[1,{\bf M}]=[{\bf x}_1,\ldots,{\bf x}_{p}]\) is the \(n\times p\) design
matrix, and \({\boldsymbol \epsilon}\sim \mathcal{N}_{n}({\bf 0},\sigma^2{\bf I}_{n})\). Note that as \(p>n\), \({\bf X}\) is not full rank.

From \eqref{eq:1}, the likelihood function is given as
\[
{\bf y}|{\boldsymbol \beta}, \sigma^2 \sim \mathcal{N}({\bf X}{\boldsymbol \beta}, \sigma^2I_n).
\]
We define the prior as follows:
\[
\beta_j|\sigma^2,\gamma_j \stackrel{ind}{\sim} (1-\gamma_j) \mathcal{N}(0, \sigma^2\nu_0) + \gamma_j\mathcal{N}(0,\sigma^2\nu_1),
\]
where \(\nu_0\) and \(\nu_1\) will be chosen to be small and large, respectively. Note that the likelihood is independent of \({\boldsymbol \gamma}=(\gamma_1,\ldots,\gamma_p)\).
Assume
\[
\sigma^2 \sim \mathcal{IG}(\frac{a}{2}, \frac{b}{2}),
\]
which is also independent of \({\boldsymbol \gamma}\). We consider
\[
\gamma_j \stackrel{iid}{\sim} Ber(\omega).
\]

To make our model robust to the choice of \(\omega\), we will assign the following prior on \(\omega\).
\[w\sim \mathcal{B}(c_1,c_2),\]
where we will use \(c_1=c_2=1\), which leads to the uniform distribution. Recall the density function of beta distribution is proportional \(\pi(w)\propto w^{c_1-1}(1-w)^{c_2-1}\).

It is easy to show that the \textbf{full conditionals} are as follows:

\begin{itemize}
\item
  \begin{enumerate}
  \def\labelenumi{\arabic{enumi})}
  \tightlist
  \item
    \[\beta_j|{\boldsymbol \beta}_{-j}, \sigma^2, {\boldsymbol \gamma}, {\bf y}\sim \mathcal{N}(\tilde{\beta}_j, \frac{\sigma^2}{\mu_j}).\]
    where \(\mu_j= x_j^{{\top}}x_j + \frac{1}{\nu_{\gamma_j}}\) and \(\tilde{\beta}_j = \mu_j^{-1}x_j^{{\top}}({\bf y}- {\bf X}_{-j}{\boldsymbol \beta}_{-j})\).
  \end{enumerate}
\item
  \begin{enumerate}
  \def\labelenumi{\arabic{enumi})}
  \setcounter{enumi}{1}
  \tightlist
  \item
    \[
    \gamma_j |{\boldsymbol \gamma}_{-j}, {\boldsymbol \beta}, \sigma^2, {\bf y}\sim Ber\left(\frac{ \nu_1^{-\frac{1}{2}}
    \exp\left(-\frac{1}{2\sigma^2\nu_1}\beta_j^2\right)\omega}
    {  \nu_0^{-\frac{1}{2}}\exp\left(-\frac{1}{2\sigma^2\nu_0}\beta_j^2\right)\left(1-\omega\right)+
     \nu_1^{-\frac{1}{2}}\exp\left(-\frac{1}{2\sigma^2\nu_1}\beta_j^2\right)\omega}\right).
    \]
  \end{enumerate}
\item
  \begin{enumerate}
  \def\labelenumi{\arabic{enumi})}
  \setcounter{enumi}{2}
  \tightlist
  \item
    \[\sigma^2|{\boldsymbol \beta}, {\boldsymbol \gamma}, {\bf y}\sim \mathcal{IG}(a^*,b^*),\]
    where \(a^*=\frac{1}{2}(n+p+a)\) and \(b^* = \frac{1}{2}\left(\|{\bf y}-{\bf X}{\boldsymbol \beta}\|^2 + \sum_{j=1}^{p}\frac{\beta_j^2}{\nu_{\gamma_j}} + b\right).\)
  \end{enumerate}
\item
  \begin{enumerate}
  \def\labelenumi{\arabic{enumi})}
  \setcounter{enumi}{3}
  \tightlist
  \item
    \[w|{\boldsymbol \beta},{\boldsymbol \gamma},\sigma^2,{\bf y}\sim \mathcal{B}\left(\sum_{j=1}^p \gamma_j+c_1,p-\sum_{j=1}^p \gamma_j+c_2\right).\]
  \end{enumerate}
\end{itemize}

To speed up, we consider the following conditionals:

\begin{itemize}
\item
  1')
  \[{\boldsymbol \beta}|\sigma^2, {\boldsymbol \gamma}, {\bf y}\sim \mathcal{N}(\tilde{{\boldsymbol \beta}}, {\sigma^2}({\bf X}^{\top}{\bf X}+{\bf V}^{-1}_{\boldsymbol \gamma})^{-1}),\]
  where \({\bf V}_{\boldsymbol \gamma}={\rm diag}(v_{\gamma_j})_{j=0}^p\) and \(\tilde{{\boldsymbol \beta}} = ({\bf X}^{\top}{\bf X}+{\bf V}^{-1}_{\boldsymbol \gamma})^{-1}{\bf X}^{\top}{\bf y}\).
\item
  2')
  \[
  {\boldsymbol \gamma}|{\boldsymbol \beta}, \sigma^2, {\bf y}\sim \prod_{j=0}^p Ber\left(\frac{ \nu_1^{-\frac{1}{2}}
    \exp\left(-\frac{1}{2\sigma^2\nu_1}\beta_j^2\right)\omega}
  {  \nu_0^{-\frac{1}{2}}\exp\left(-\frac{1}{2\sigma^2\nu_0}\beta_j^2\right)\left(1-\omega\right)+
   \nu_1^{-\frac{1}{2}}\exp\left(-\frac{1}{2\sigma^2\nu_1}\beta_j^2\right)\omega}\right).
  \]
\item
  3')
  \[\sigma^2|{\boldsymbol \beta}, {\boldsymbol \gamma}, {\bf y}\sim \mathcal{IG}(a^*,b^*).\]
  where \(a^*=\frac{1}{2}(n+p+a)\) and
  \(b^* = \frac{1}{2}\left(\|{\bf y}-{\bf X}{\boldsymbol \beta}\|^2 + \sum_{j=1}^{p}\frac{\beta_j^2}{\nu_{\gamma_j}} + b\right).\)
\item
  4')
  \[w|{\boldsymbol \beta},{\boldsymbol \gamma},\sigma^2,{\bf y}\sim \mathcal{B}\left(\sum_{j=1}^p \gamma_j+c_1,p-\sum_{j=1}^p \gamma_j+c_2\right).\]
\end{itemize}

\hypertarget{rcode}{%
\section{Rcode}\label{rcode}}

\begin{Shaded}
\begin{Highlighting}[]
\KeywordTok{library}\NormalTok{(invgamma)}
\NormalTok{p <-}\StringTok{ }\DecValTok{100}
\NormalTok{n =}\StringTok{ }\DecValTok{100}
\NormalTok{power <-}\StringTok{ }\KeywordTok{numeric}\NormalTok{()}
\NormalTok{e<-}\KeywordTok{rnorm}\NormalTok{(n, }\DataTypeTok{mean =} \DecValTok{0}\NormalTok{, }\DataTypeTok{sd =} \KeywordTok{sqrt}\NormalTok{(}\DecValTok{2}\NormalTok{)) }\CommentTok{# error}
\NormalTok{X<-}\KeywordTok{matrix}\NormalTok{(}\DataTypeTok{data =} \KeywordTok{rnorm}\NormalTok{(n }\OperatorTok{*}\StringTok{ }\NormalTok{p, }\DecValTok{0}\NormalTok{, }\DecValTok{1}\NormalTok{), }\DataTypeTok{nrow =}\NormalTok{ n, }\DataTypeTok{ncol =}\NormalTok{ p)}
\NormalTok{Beta <-}\StringTok{ }\KeywordTok{c}\NormalTok{(}\DecValTok{1}\NormalTok{, }\DecValTok{2}\NormalTok{, }\KeywordTok{rep}\NormalTok{(}\DecValTok{0}\NormalTok{,(p }\OperatorTok{-}\StringTok{ }\DecValTok{3}\NormalTok{)), }\DecValTok{3}\NormalTok{)}\CommentTok{# exclude intercept beta0}
\NormalTok{y <-}\StringTok{ }\NormalTok{X }\OperatorTok\StringTok{ }\NormalTok{Beta }\OperatorTok{+}\StringTok{ }\NormalTok{e }\CommentTok{# true model}
\NormalTok{true.gamma<-}\KeywordTok{as.numeric}\NormalTok{(Beta }\OperatorTok{!=}\StringTok{ }\DecValTok{0}\NormalTok{)}
\CommentTok{#setup for initial values#####}
\NormalTok{hat.beta <-}\StringTok{ }\KeywordTok{as.numeric}\NormalTok{(}\KeywordTok{solve}\NormalTok{(}\KeywordTok{t}\NormalTok{(X) }\OperatorTok\StringTok{ }\NormalTok{(X) }\OperatorTok{+}\StringTok{ }\KeywordTok{diag}\NormalTok{(}\DecValTok{1}\NormalTok{, p)) }\OperatorTok\StringTok{ }\KeywordTok{t}\NormalTok{(X) }\OperatorTok\StringTok{ }\NormalTok{y) }\CommentTok{#p-dim vector}
\NormalTok{hat.gamma <-}\StringTok{ }\KeywordTok{rep}\NormalTok{(}\DecValTok{1}\NormalTok{, p)}
\NormalTok{hat.sig2 <-}\StringTok{ }\KeywordTok{mean}\NormalTok{((y }\OperatorTok{-}\StringTok{ }\NormalTok{X }\OperatorTok\StringTok{ }\NormalTok{hat.beta)}\OperatorTok{^}\DecValTok{2}\NormalTok{)}
\CommentTok{#setup for priors ############}
\NormalTok{w <-}\StringTok{ }\FloatTok{0.5}
\NormalTok{v0 <-}\StringTok{ }\FloatTok{0.001}
\NormalTok{v1 <-}\StringTok{ }\DecValTok{1000}
\NormalTok{v01 <-}\StringTok{ }\KeywordTok{c}\NormalTok{(v0, v1)}
\NormalTok{a0 <-}\StringTok{ }\DecValTok{1}
\NormalTok{b0 <-}\StringTok{ }\DecValTok{1}
\CommentTok{###############################}
\NormalTok{MC.size <-}\StringTok{ }\DecValTok{2000} \OperatorTok{+}\StringTok{ }\DecValTok{3000}
\NormalTok{hat.BETA <-}\StringTok{ }\KeywordTok{matrix}\NormalTok{(}\DecValTok{0}\NormalTok{, MC.size, p) }\CommentTok{# to store beta for each iteration}
\NormalTok{hat.Gamma <-}\StringTok{ }\KeywordTok{matrix}\NormalTok{(}\DecValTok{0}\NormalTok{, MC.size, p) }\CommentTok{# to store z for each iteration}
\NormalTok{hat.Sig2 <-}\StringTok{ }\KeywordTok{rep}\NormalTok{(}\DecValTok{0}\NormalTok{, MC.size) }\CommentTok{# to store variance for each iteration}
\ControlFlowTok{for}\NormalTok{ (goh }\ControlFlowTok{in} \DecValTok{1}\OperatorTok{:}\NormalTok{MC.size) \{}
  \CommentTok{# Gibbs sampling }
  \CommentTok{# 1) for beta_j}
  \ControlFlowTok{for}\NormalTok{ (j }\ControlFlowTok{in} \DecValTok{1}\OperatorTok{:}\NormalTok{p) \{}
\NormalTok{    mu_j <-}\StringTok{ }\KeywordTok{t}\NormalTok{(X[, j]) }\OperatorTok\StringTok{ }\NormalTok{X[, j] }\OperatorTok{+}\StringTok{ }\DecValTok{1}\OperatorTok{/}\NormalTok{v01[(hat.gamma[j] }\OperatorTok{+}\StringTok{ }\DecValTok{1}\NormalTok{)]}
\NormalTok{    y.star <-}\StringTok{ }\NormalTok{y }\OperatorTok{-}\StringTok{ }\NormalTok{X[, }\OperatorTok{-}\NormalTok{j] }\OperatorTok\StringTok{ }\NormalTok{hat.beta[}\OperatorTok{-}\NormalTok{j]}
\NormalTok{    tilde.beta.j <-}\StringTok{ }\KeywordTok{as.numeric}\NormalTok{((}\DecValTok{1}\OperatorTok{/}\NormalTok{mu_j) }\OperatorTok{*}\StringTok{ }\KeywordTok{t}\NormalTok{(X[, j]) }\OperatorTok\StringTok{ }\NormalTok{y.star)}
\NormalTok{    var.beta <-}\StringTok{ }\KeywordTok{as.numeric}\NormalTok{(hat.sig2}\OperatorTok{/}\NormalTok{mu_j)}
\NormalTok{    hat.beta[j] <-}\StringTok{ }\KeywordTok{rnorm}\NormalTok{(}\DecValTok{1}\NormalTok{, tilde.beta.j, }\KeywordTok{sqrt}\NormalTok{(var.beta))  }\CommentTok{#sampling from beta_j|others}
\NormalTok{  \}}
  \CommentTok{# 2) gamma_j}
\NormalTok{  p.j <-}\StringTok{ }\KeywordTok{dnorm}\NormalTok{(hat.beta, }\DecValTok{0}\NormalTok{, }\KeywordTok{sqrt}\NormalTok{(v1 }\OperatorTok{*}\StringTok{ }\NormalTok{hat.sig2)) }\OperatorTok{*}\StringTok{ }\NormalTok{w}
\NormalTok{  q.j <-}\StringTok{ }\KeywordTok{dnorm}\NormalTok{(hat.beta, }\DecValTok{0}\NormalTok{, }\KeywordTok{sqrt}\NormalTok{(v0 }\OperatorTok{*}\StringTok{ }\NormalTok{hat.sig2)) }\OperatorTok{*}\StringTok{ }\NormalTok{(}\DecValTok{1} \OperatorTok{-}\StringTok{ }\NormalTok{w)}
\NormalTok{  prob.j <-}\StringTok{ }\NormalTok{p.j}\OperatorTok{/}\NormalTok{(p.j }\OperatorTok{+}\StringTok{ }\NormalTok{q.j)}
\NormalTok{  hat.gamma <-}\StringTok{ }\KeywordTok{rbinom}\NormalTok{(p, }\DecValTok{1}\NormalTok{, prob.j)}
\NormalTok{  hat.Gamma[goh, ] <-}\StringTok{ }\NormalTok{hat.gamma}
\NormalTok{  hat.BETA[goh, ] <-}\StringTok{ }\NormalTok{hat.beta}
  \CommentTok{# 3) sig2}
\NormalTok{  a.star <-}\StringTok{ }\DecValTok{1}\OperatorTok{/}\DecValTok{2} \OperatorTok{*}\StringTok{ }\NormalTok{(n }\OperatorTok{+}\StringTok{ }\NormalTok{p }\OperatorTok{+}\StringTok{ }\NormalTok{a0)}
\NormalTok{  v.z_j <-}\StringTok{ }\NormalTok{hat.gamma }\OperatorTok{*}\StringTok{ }\NormalTok{v1 }\OperatorTok{+}\StringTok{ }\NormalTok{(}\DecValTok{1} \OperatorTok{-}\StringTok{ }\NormalTok{hat.gamma) }\OperatorTok{*}\StringTok{ }\NormalTok{v0}
\NormalTok{  b.star <-}\StringTok{ }\DecValTok{1}\OperatorTok{/}\DecValTok{2} \OperatorTok{*}\StringTok{ }\NormalTok{(}\KeywordTok{sum}\NormalTok{((y }\OperatorTok{-}\StringTok{ }\NormalTok{X }\OperatorTok\StringTok{ }\NormalTok{hat.beta)}\OperatorTok{^}\DecValTok{2}\NormalTok{) }\OperatorTok{+}\StringTok{ }\KeywordTok{sum}\NormalTok{(hat.beta}\OperatorTok{^}\DecValTok{2}\OperatorTok{/}\NormalTok{v.z_j) }\OperatorTok{+}\StringTok{ }\NormalTok{b0)}
\NormalTok{  hat.sig2 <-}\StringTok{ }\KeywordTok{rinvgamma}\NormalTok{(}\DecValTok{1}\NormalTok{, }\DataTypeTok{shape =}\NormalTok{ a.star, }\DataTypeTok{rate =}\NormalTok{ b.star)}
  \KeywordTok{par}\NormalTok{(}\DataTypeTok{mfrow=}\KeywordTok{c}\NormalTok{(}\DecValTok{1}\NormalTok{,}\DecValTok{1}\NormalTok{))}
  \KeywordTok{plot}\NormalTok{(hat.gamma, }\DataTypeTok{main =} \KeywordTok{paste}\NormalTok{(}\StringTok{"rep:"}\NormalTok{, goh))}
  \KeywordTok{points}\NormalTok{(true.gamma, }\DataTypeTok{col =} \DecValTok{2}\NormalTok{, }\DataTypeTok{pch =} \StringTok{"*"}\NormalTok{)}
\NormalTok{\}}
\KeywordTok{colMeans}\NormalTok{(hat.Gamma)}\OperatorTok{>}\FloatTok{0.5}
\end{Highlighting}
\end{Shaded}

\hypertarget{appendix}{%
\section{Appendix}\label{appendix}}

\begin{itemize}
\tightlist
\item
  For \(\beta_j|{\boldsymbol \beta}_{-j}, \sigma^2, {\boldsymbol \gamma}, {\bf y}\) \((j = 1,2, \ldots,p)\), we have
\end{itemize}

\begin{eqnarray*}
\pi(\beta_j|{\boldsymbol \beta}_{-j}, \sigma^2, {\boldsymbol \gamma}, {\bf y})&\propto&f({\bf y}|{\boldsymbol \beta}, \sigma^2)\pi({\boldsymbol \beta}|{\boldsymbol \gamma}, \sigma^2)\\
&=&f({\bf y}|{\boldsymbol \beta}_{{\boldsymbol \gamma}}, \sigma^2) \prod_{k=1}^{p}\pi(\beta_k|Z_k, \sigma^2)\\
&\propto&f({\bf y}|{\boldsymbol \beta}, \sigma^2)\pi(\beta_j|\gamma_j, \sigma^2)\\
&\propto& \exp\left(-\frac{1}{2\sigma^2}\|{\bf y}-{\bf X}{\boldsymbol \beta}\|^2 \right)
\exp\left(-\frac{1}{2\sigma^2\nu_{\gamma_j}}\beta_j^2\right)\\
&=& \exp\left(-\frac{1}{2\sigma^2}\|{\bf y}- {\bf X}_{-j}{\boldsymbol \beta}_{-j} - x_j \beta_j\|^2\right)
\exp\left(-\frac{1}{2\sigma^2\nu_{\gamma_j}}\beta_j^2\right)\\
&=& \exp\left(-\frac{1}{2\sigma^2}\|{\bf y}^* - x_j \beta_j\|^2\right)
\exp\left(-\frac{1}{2\sigma^2\nu_{\gamma_j}}\beta_j^2\right)\\
&=&\exp\left[-\frac{1}{2\sigma^2}\left({{\bf y}^*}^{{\top}}{\bf y}^* - 2\beta_j x_j^{{\top}}{\bf y}^*
+ \beta_j x_j^{{\top}}x_j\beta_j + \frac{1}{\nu_{\gamma_j}}\beta_j^2\right)\right]\\
&\propto&\exp\left[-\frac{1}{2\sigma^2}\left(\beta_j^2(x_j^{{\top}}x_j +
\frac{1}{\nu_{\gamma_j}}) - 2\beta_j x_j^{{\top}}{\bf y}^* \right)\right]\\
&=&\exp\left[-\frac{1}{2\sigma^2}\left(\beta_j^2(x_j^{{\top}}x_j +
\frac{1}{\nu_{\gamma_j}}) - 2\beta_j (x_j^{{\top}}x_j + \frac{1}{\nu_{\gamma_j}})(x_j^{{\top}}x_j +
\frac{1}{\nu_{\gamma_j}})^{-1} x_j^{{\top}}{\bf y}^* \right)\right]\\
&=& \exp\left[-\frac{a}{2\sigma^2}\left(\beta_j^2 - 2a \beta_j \tilde \beta_j\right)\right]\\
&\propto& \exp\left[-\frac{a}{2\sigma^2}\left(\beta_j -\tilde \beta_j\right)^2\right]\\
\end{eqnarray*}

where \({\bf y}^* = {\bf y}- {\bf X}_{-j}{\boldsymbol \beta}_{-j}, \quad \mu_j= x_j^{{\top}}x_j + \frac{1}{\nu_{\gamma_j}},\quad \tilde \beta_j = \mu_j^{-1}x_j^{{\top}}{\bf y}^*\).
Hence, \[\beta_j|{\boldsymbol \beta}_{-j}, \sigma^2, {\boldsymbol \gamma}, {\bf y}\sim \mathcal{N}(\tilde \beta_j, \frac{\sigma^2}{\mu_j}).\]

\begin{itemize}
\tightlist
\item
  For \(\pi(\gamma_j|{\boldsymbol \gamma}_{-j}, {\boldsymbol \beta}, \sigma^2, {\bf y})\) \((j = 1,2, \ldots,p)\) we have
\end{itemize}

\begin{eqnarray*}
\pi(\gamma_j|{\boldsymbol \gamma}_{-j}, {\boldsymbol \beta}, \sigma^2, {\bf y})&\propto& \pi({\boldsymbol \gamma},{\boldsymbol \beta}|\sigma^2)\\
&=&\pi({\boldsymbol \beta}|\sigma^2, {\boldsymbol \gamma})\pi({\boldsymbol \gamma})\\
&=& \prod_{k=1}^{p}\pi(\beta_{k}|\sigma^2, z_{k})\prod_{i=1}^{p}\pi(z_i)\\
&\propto&\pi(\beta_{j}|\sigma^2, z_{j})\pi(\gamma_j)\\
&\propto&\nu_{\gamma_j}^{-\frac{1}{2}}\exp\left(-\frac{1}{2\sigma^2\nu_{\gamma_j}}\beta_j^2\right)\omega^{\gamma_j}
(1-\omega)^{1-\gamma_j}.
\end{eqnarray*}

Note that

\begin{eqnarray*}
\pi(\gamma_j = 0|{\boldsymbol \gamma}_{-j}, {\boldsymbol \beta}, \sigma^2, {\bf y})&=& C \nu_0^{-\frac{1}{2}}\exp\left(-\frac{1}{2\sigma^2\nu_0}\beta_j^2\right)\left(1-\omega\right);\\
\pi(\gamma_j = 1|{\boldsymbol \gamma}_{-j}, {\boldsymbol \beta}, \sigma^2, {\bf y})&=& C \nu_1^{-\frac{1}{2}}\exp\left(-\frac{1}{2\sigma^2\nu_1}\beta_j^2\right)\omega.
\end{eqnarray*}

This implies that

\begin{eqnarray*}
\pi(\gamma_j = 1|{\boldsymbol \gamma}_{-j}, {\boldsymbol \beta}, \sigma^2, {\bf y})
&=&\frac{P(\gamma_j = 1,\Omega)}{\sum_{\gamma_j}P(\gamma_j,\Omega)}\\
&=& \frac{C \nu_1^{-\frac{1}{2}}\exp\left(-\frac{1}{2\sigma^2\nu_1}\beta_j^2\right)\omega}
{ C \nu_0^{-\frac{1}{2}}\exp\left(-\frac{1}{2\sigma^2\nu_0}\beta_j^2\right)\left(1-\omega\right)+
C \nu_1^{-\frac{1}{2}}\exp\left(-\frac{1}{2\sigma^2\nu_1}\beta_j^2\right)\omega}\\
&=& \frac{ \nu_1^{-\frac{1}{2}}\exp\left(-\frac{1}{2\sigma^2\nu_1}\beta_j^2\right)\omega}
{  \nu_0^{-\frac{1}{2}}\exp\left(-\frac{1}{2\sigma^2\nu_0}\beta_j^2\right)\left(1-\omega\right)+
 \nu_1^{-\frac{1}{2}}\exp\left(-\frac{1}{2\sigma^2\nu_1}\beta_j^2\right)\omega}.
\end{eqnarray*}

Hence,
\[
\gamma_j |{\boldsymbol \gamma}_{-j}, {\boldsymbol \beta}, \sigma^2, {\bf y}\sim Ber\left(\frac{ \nu_1^{-\frac{1}{2}}
    \exp\left(-\frac{1}{2\sigma^2\nu_1}\beta_j^2\right)\omega}
{  \nu_0^{-\frac{1}{2}}\exp\left(-\frac{1}{2\sigma^2\nu_0}\beta_j^2\right)\left(1-\omega\right)+
 \nu_1^{-\frac{1}{2}}\exp\left(-\frac{1}{2\sigma^2\nu_1}\beta_j^2\right)\omega}\right).
\]

\begin{itemize}
\tightlist
\item
  For \(\sigma^2|{\boldsymbol \beta}, {\boldsymbol \gamma}, {\bf y}\), we have
\end{itemize}

\begin{eqnarray*}
\pi(\sigma^2|{\boldsymbol \beta}, {\boldsymbol \gamma}, {\bf y})&\propto&f({\bf y}|{\boldsymbol \beta},\sigma^2,{\boldsymbol \gamma})\pi({\boldsymbol \beta},\sigma^2|{\boldsymbol \gamma})\\
&=&f({\bf y}|{\boldsymbol \beta}, \sigma^2, {\boldsymbol \gamma})\pi({\boldsymbol \beta}|\sigma^2,{\boldsymbol \gamma})\pi(\sigma^2|{\boldsymbol \gamma})\\
&=&f({\bf y}|{\boldsymbol \beta}, \sigma^2)\pi({\boldsymbol \beta}|\sigma^2,{\boldsymbol \gamma})\pi(\sigma^2)\\
&\propto&(\sigma^2)^{-\frac{n}{2}}\exp\left(-\frac{1}{2\sigma^2}\|{\bf y}-{\bf X}{\boldsymbol \beta}\|^2\right)\times
\prod_{j=1}^{p}\left[(\sigma^2\nu_{\gamma_j})^{-\frac{1}{2}}\exp\left(-\frac{1}{2\sigma^2\nu_{\gamma_j}}
\beta_j^2\right)\right]\\
&&\times(\sigma^2)^{-\frac{a}{2}-1}\exp\left(-\frac{b}{2\sigma^2}\right)\\
&\propto&(\sigma^2)^{-\frac{n}{2}}\exp\left(-\frac{1}{2\sigma^2}\|{\bf y}-{\bf X}{\boldsymbol \beta}\|^2\right)\times
(\sigma^2)^{-\frac{p}{2}}\exp\left(-\frac{1}{2\sigma^2}\sum_{j=1}^{p}\frac{\beta_j^2}{\nu_{\gamma_j}}\right)\\
&&\times(\sigma^2)^{-\frac{a}{2}-1}\exp\left(-\frac{b}{2\sigma^2}\right)\\
&=& (\sigma^2)^{-\frac{1}{2}(n+p+a)-1}\exp\left(-\frac{\frac{1}{2}\left(\|{\bf y}-{\bf X}{\boldsymbol \beta}
\|^2 + \sum_{j=1}^{p}\frac{\beta_j^2}{\nu_{\gamma_j}} + b\right)}{\sigma^2}\right)\\
&=&(\sigma^2)^{-a^* -1}\exp(-\frac{b^*}{\sigma^2}),
\end{eqnarray*}

where \(a^*=\frac{1}{2}(n+p+a)\) and \(b^* = \frac{1}{2}\left(\|{\bf y}-{\bf X}{\boldsymbol \beta}\|^2 + \sum_{j=1}^{p}\frac{\beta_j^2}{\nu_{\gamma_j}} + b\right).\)

Therefore,
\[\sigma^2|{\boldsymbol \beta}, {\boldsymbol \gamma}, {\bf y}\sim \mathcal{IG}(a^*,b^*).\]

\begin{itemize}
\tightlist
\item
  For \(w|{\boldsymbol \beta},{\boldsymbol \gamma},\sigma^2,{\bf y}\), we have
\end{itemize}

\begin{eqnarray*}
\pi(w|{\boldsymbol \beta},{\boldsymbol \gamma},\sigma^2,{\bf y})&\propto& \pi({\boldsymbol \gamma}|w) \pi(w)\\
&\propto&\left[ \prod_{j=1}^p w^{\gamma_j}(1-w)^{1-\gamma_j}\right] w^{c_1-1}(1-w)^{c_2-1}\\
&\propto& w^{\sum_{j=1}^p \gamma_j+c_1-1}(1-w)^{p-\sum_{j=1}^p \gamma_j+c_2-1}.
\end{eqnarray*}

We therefore have
\[w|{\boldsymbol \beta},{\boldsymbol \gamma},\sigma^2,{\bf y}\sim \mathcal{B}\left(\sum_{j=1}^p \gamma_j+c_1,p-\sum_{j=1}^p \gamma_j+c_2\right).\]

\bibliography{book.bib}


\end{document}
